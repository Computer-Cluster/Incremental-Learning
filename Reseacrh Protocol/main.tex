\documentclass[10pt,a4paper]{article}
\usepackage[utf8]{inputenc}
\usepackage[spanish,mexico]{babel} 
\usepackage{amsfonts}
\usepackage{amsmath}
\usepackage{amssymb}
\usepackage{multicol}
\usepackage{graphicx}
\usepackage{subcaption}
\usepackage{float}
\usepackage{lipsum}
\usepackage{xcolor}
\usepackage[rightcaption]{sidecap}
\usepackage{array}
\usepackage{framed}
\usepackage{color}
\usepackage{wrapfig}
\definecolor{shadecolor}{RGB}{224,238,238}
\usepackage{listings}
\usepackage[hidelinks]{hyperref}
\usepackage{pdfpages}
\usepackage{pdflscape}
\usepackage{verbatim}
\pagestyle{empty}
\usepackage[nottoc]{tocbibind}
\usepackage{times}
%\usepackage{apacite}

\graphicspath{ {images/} }

% Márgenes
\usepackage[top=2.5cm, bottom=2.5cm, left=3.0cm, right=2.5cm]{geometry}

\title{Aprendizaje incremental para la tarea de reconocimiento
de dígitos con Redes Neuronales Artificiales}

\makeindex

\begin{document}

    \includepdf{textFiles/portada.pdf}
    \newpage
    \tableofcontents
    \listoffigures
    \newpage
    \maketitle
    \begin{abstract}
        El aprendizaje incremental es un área de la Inteligencia Artificial la cual permite agregar nuevo conocimiento 
        a un modelo (e.g. Redes Neuronales Artificiales) sin la necesidad de entrenar el modelo con toda la información 
        histórica de la tarea en cuestión \cite{bullinaria2009}. En el presente trabajo de investigación se ocupará el modelo de Redes 
        Neuronales Artificiales enfocada en la clasificación de dígitos escritos a mano usando el algoritmo de 
        entrenamiento de backpropagation, con redes Multi capa Perceptron y duplicación de pesos múltiples 
        simulando memoria a corto y largo plazo para mejorar los resultados presentados en \cite{bullinaria2009}.\\

        \textbf{Palabras claves:} Aprendizaje Incremental, Redes Multi Capa, Redes Neuronales Artificiales
    \end{abstract}

    \section{Introduccion}
Se requiere el uso de las redes neuronales para poder dar predicciones de sucesos los cuales pasan en nuestras vidas, al momento de decir predicción lo primero que se 
nos viene a la mente es la astrología, pero no todo es astrología, nosotros tenemos el poder de realizar predicciones con el uso de tecnología. En Microsoft los del 
departamento de AI son conocidos como \textbf{AIWizzard}, ya que nosotros fabricamos magia, pero que pasa si implementamos esta magia en la vida de un estudiante o en la vida de 
un profesional,  claramente vamos a poder observar un cambio enorme en el cual su desempeño va a mejorar ya que conocerá, aprenderá, mejorará y optimizará todos sus labores. \\
En especial el uso de las redes neuronales es un campo importante en la AI el cual nos va a permitir mejoras gigantes en nuestro mundo.

El aprendizaje incrementado es un método el cual a sido implementado en el área de la inteligencia artificial, ya que al realizar tareas de inteligencia artificial especificas esta rama nos 
ayuda a optimizarla para que el algoritmo sea más eficiente
    \section{Planeamiento}

    Indicar de forma general como es el funcionamiento de las redes neuronales en el uso general, como es que se produce la falta de memoria en 
    estos modelos de predicciones. Poner en pr\'actica los modelos de experimentación de John Bullinaria para comprobarlos con
    problematicas m\'as robustas y optimizarlo para que sea un m\'etodo eficiente.
    
\section{Objetivos}
    Diseñar una red neuronal para aprendizaje incremental basada en el principio de la memoria a corto y largo plazo, buscando usar más de dos categorías para el reconocimiento de dígitos.
    \subsection{Objetivos Particulares}
        \begin{enumerate}
            \item Implementar el algoritmo de John A. Bullinaria para el reconocimiento de dígitos con aprendizaje a corto y largo plazo con los parámetros que él indica.
            \item Obtener el conjunto de datos de Optical Digits y analizar que esté conforme al artículo de John A. Bullinaria.
            \item Separar el conjunto de entrenamiento y de prueba de acuerdo a lo que se explica en el artículo base y hacer los experimentos para que se obtengan los mismo resultados.
            \item Programar y modificar el algoritmo base, hacer que sean más de dos nodos, y repartir las tazas de aprendizaje proporcionalmente.
            \item Probar la nueva implementación con el mismo conjunto de datos y ver si hay una diferencia significativa.
        \end{enumerate}
\section{Justificaci\'on}

        Las redes neuronales apoyan a la resoluci\'on de distintos problemas, pero el Maching Learning 
        tiene una deficiencia que es al momento de aumentar los datos a dichos modelos, la deficiencia 
        que se obtiene es enorme que causa que los proyectos sean obsoleto \cite{Bullinaria2009}. Los 
        resultados que se han obtenido no funcionan a la perfección, la memoria a corto plazo olvida poco 
        pero va olvidando, y lo ideal sería que no olvidara, biológicamente nosotros no podemos 
        hacer muchas modificaciones por lo mismo que implica, pero computacionalmente nada puede 
        impedir que se pruebe con más configuraciones y llegar al punto en donde toda la información 
        que llega se acumule y si no hay problema de almacenamiento que se siga acumulando y que no olvide, 
        eso podría ser bueno en algunas situaciones.

        Hacer este trabajo puede hacer que funcionen mejor las técnicas, pues ahorraría más energía 
        en lugar de hacer entrenamientos muy grandes cada determinado tiempo e incluir todos los datos 
        pasados de forma paulatina. Así como el ahorro de tiempo, de procesamiento, los tiempos de 
        entrenamientos se reducirían y se reduciría la pérdida de información.

        Existe una gran variedad de herramientas las cuales nos permiten codificar una red neuronal, existen 2 empresas 
        importantes las cuales nos prestan sus servicios, las cuales son:
        \begin{enumerate}
            \item Microsoft.
            \item Google.
        \end{enumerate}
        Por el lado de Microsoft tenemos lo que es la plataforma de Azure que nos renta una maquina virtual donde podemos 
        pogramar en python.

        Del lado de google tenemos lo que es google Colab que igual nos brinda una maquina virtual para realizar experimentos de Maching
        Learning, la \'unica diferencia a Azure es que es gratuito, aqu\'i también se puede programar en Python.

        Como se observa en los dos se puede programar en pyhton y es porque este lenguaje es una herramienta de 
        software libre que no requiere licencia, es relativamente fácil poder depurar un código y permite acelerar 
        más el desarrollo de aplicaciones,  a diferencia de otros lenguajes más estructurados 
        como c o java, adem\'as tiene m\'as librerias para el desarrollo de Maching Learning. \\

        TensorFlow es una libreria de python que te permite construir y entrenar redes neuronales para detectar patrones y
        razonamientos usados por los humanos. \\

        Keras es un framework de alto nivel para el aprendizaje, escrito en Python y capaz de correr sobre los 
        frameworks TensorFlow. Fue desarrollado con el objeto de facilitar un proceso de experimentación rápida. Diseñado para construir por bloques la arquitectura de cada red neuronal, incluyendo redes convolucionales y modelos recurrentes, que son las que permiten, junto a los bloques “más tradicionales”, entrenar aprendizaje profundo.


\section{Delimitación}
            
    En la siguiente investigación solamente se van a utilizar redes neuronales artificiales, cabe mencionar que este no es el único tipo de red, porque también tenemos lo que es \cite{royo2021}:
    \begin{itemize}
        \item Redes Neuronales Monocapa.
        \item Redes Neuronales Perceptrón Multicapa (MLP).
        \item Redes Neuronales Convulcionales (CNN).
        \item Redes Neuronales Recurrentes (RNN).
        \item Redes de Base Radial (RBF).
    \end{itemize}
    Pero para este experimento vamos a utilizar ANN que es el algoritmo que por el momento nos beneficiaría, cabe mencionar que no usaremos algoritmos geneticos, ya que si se implenta, se estara
    optimizando y el objetivo principal es utilizar el aprendizaje incrementado para que acepte más datos de entrenamiento.

\section{Consecuencias}

    Si el experimento funciona a la perfecci\'on ocurrir\'a lo siguiente:
    \begin{enumerate}
        \item Habra menos olvido.
        \item Los procesos tardaran menos tiempo.
    \end{enumerate}

    Esto sucedera porque se van a poder ingresar m\'as datos a nuestro almacen sin 
    tener que volver a codificar nuestro modelo.\\
    Los modelos de predicci\'on van a ser m\'as precisos, porque se podran ingresar 
    datos mensualmente o hasta semanalmente, esto provocara que el proyecto este trabajando 
    con datos actuales.

    Pueden funcionar para proyectos y predicciones tan simples tanto la preiddci\'on climatol\'ogica
    hasta predicciones de la bolsa de valores y predicci\'on del bitcoin.


    \section{Objetivos}
    Diseñar una red neuronal artificial para aprendizaje incremental basada en el principio de la memoria a corto y largo plazo, buscando usar más de dos capas de pesos duplicados para el reconocimiento de dígitos, y con una menor perdida de información que trabajos previos.
    \subsection{Objetivos Particulares}
        \begin{enumerate}
            \item Implementar el algoritmo mostrado en \cite{bullinaria2009} para el reconocimiento de dígitos con aprendizaje a corto y largo plazo con los parámetros que ahí se indican.
            \item Obtener el conjunto de datos de Optical Digits, limpiar los datos y prepararlos segun lo indicado con \cite{bullinaria2009}.
            \item Separar el conjunto de entrenamiento y de prueba de acuerdo a lo que se explica en el artículo de Bullinaria yprobar el primer código implementado en miras de comprobar  los resultados previamente mostrados en \cite{bullinaria2009}.
            \item Tomando como base el algoritmo implementado,  y extenderlo para permitir mas de dos pesos duplicados, aplicando el conjunto de datos previamente mostrado.
            \item Comparar ambas implementaciones en búsca de una reducción significativa de las tasas de aprendizaje con respecto a trabajos previos en la literatura.
        \end{enumerate}
    \section{Hipotesis}
    Al tener m\'as conexiones en un modelo de red neuronal usualmente se tendr\'a mejor ajuste de los 
    problemas, ya que al momento de predecir este ser\'a mucho m\'as eficiente de manejar.
    \section{Justificaci\'on}

    Las redes neuronales permite el aprendizaje automático y la resoluci\'on 
    de distintos problemas,  pero como se comentó anteriormente,  las técnicas 
    de aprendizaje máquina, tiene una deficiencia que es al momento de aumentar 
    los nuevos bloques de datos que lelguan para aprender,  se obtiene un deteriro 
    en el rendimiento de aprendizaje de información y olvido de la información anterior
    \cite{bullinaria2009}.   (REFIVEN COMO VOY PONEINDO LA REDACCIÓN DE LO QNTERIOR, Y 
    CO NELLO TRATEN DE CAMBIAR LO QUE VIENE PARA QUE LE MOJOREN EL ESTILO, DE AQUÍ EN 
    ADELANTE SOLO LES HARE OBSERVACIONES EN ESPERA QUE USTEDES MEJOREN EN GENERAR LA 
    REDACCIÓN) Los resultados que se han obtenido no funcionan a la perfección, la 
    memoria a corto plazo olvida poco pero va olvidando, y lo ideal sería que no 
    olvidara. Biológicamente los humanos pueden aprender nuevas tareas, o información 
    nueva de un problema, y no olvida de forma siginficativa lo que anteriormente 
    aprendio, no obstante eso no pasa actualemnte co loas RNA y en general con 
    cualquier algoritmo de aprendizaje máquina, . En otro sentido los humanos 
    ya tenesmo cierta configuración en el cerebro que nos permite aprender como 
    lo hacemos actualmente,  y se peude afirmar que por elmomento no hay ningun 
    procedimiento (quirurjico o no) que permita modificar la estructura del cerebro 
    para aprender mas y olvidar menos. 

    No obstante,  computacionalmente nada puede impedir que se experimente con más 
    configuraciones y llegar al punto en donde toda la información que ingrese a un 
    modelo (e.g. RNAs) se acumule y si no hay problema de almacenamiento que se siga 
    acumulando y que no olvide, eso podría ser bueno en diversas situaciones.

    Desde el punto de vista computacional, si llega nueva información y no se ocupa aprendizaje 
    incremental, ello implicara volver a entrenar todo el sistema con la información anterior y 
    la actual (e.g. $d_{1}$ y $d_{2}$) y considerando que una de las desventajas que tienen la 
    RNAs es que el entremainento es un cuello de botella, siendo este donde se llevá la mayor 
    parte de cómputo y por consiguenote de energía. Ello implica que volver a entranar con todo 
    la infomación acumulada, gastará más energía y tiempo que si solo se entrana con la nueva 
    información que llega al modelo. En su contraparte,  existe una gran variedad de herramientas 
    las cuales permiten codificar una red neuronal artificial con librerias ya preexistente, por 
    el momento se expondrán solo 2 empresas, siendo estas las más importantes: Microsoft y Google. 
    La primera cuenta con la plataforma de Azure que nos renta una maquina virtual donde se puede 
    programar en Python.  Por el contrario, Google cuenta con Google Colab que igual nos brinda 
    una maquina virtual para realizar experimentos de Maching Learning, la \'unica diferencia 
    contra Azure es que dicha herramienta es gratuita, una similitud que tienen es que en las 
    dos se puede programar en el mismo lenguaje.

    Como se observa ambas herramientas permiten la programaci\'on en Pyhton y esto se debe a que 
    dicho lenguaje es una herramienta de software libre que no requiere licencia, es relativamente 
    fácil poder depurar un código y permite acelerar más el desarrollo de aplicaciones,  a diferencia 
    de otros lenguajes más estructurados como C o Java, adem\'as tiene m\'as librerías para el 
    desarrollo de Maching Learning e.g TensorFlow, Numpi, entre otras. \\

    TensorFlow es una librería de Python que permite construir y entrenar redes neuronales para 
    detectar patrones y razonamientos usados por los humanos, en la presente investigaci\'on se 
    usar\'a dado a que favorece la creaci\'on de una RNA, permite la elaboraci\'on de cualquier 
    tipo de algoritmo de Machine Learning, cabe mencionar que también se puede usar para Deep Learning, 
    facilita la adquisici\'on de datos modelos de capacitaci\'on, predicciones y refinamiento de 
    resultados, esta disponible para el uso en computadores personales, pero es recomendado 
    usarlo en su propio editor en la nube que es Colab. \\ 

    Keras (MISMO COMENTARIO QUE LES PONGO DE TENSORFLOW)es un framework de alto nivel para 
    el aprendizaje, escrito en Python y capaz de correr sobre los frameworks TensorFlow. Fue 
    desarrollado con el objeto de facilitar un proceso de experimentación rápida. Diseñado 
    para construir por bloques la arquitectura de cada red neuronal, incluyendo redes 
    convolucionales y modelos recurrentes, que son las que permiten, junto a los bloques 
    "más tradicionales", entrenar aprendizaje profundo.
    \section{Delimitación}
\label{sec:delimitation}

    En la siguiente investigación solamente se van a utilizar redes neuronales artificiales, 
    cabe mencionar que este no es el único tipo de red, porque también se tiene lo que son Redes Neuronales Monocapa,
    Redes Neuronales Perceptrón Multicapa (RNPM), Redes Neuronales Convulcionales (RNC), Redes Neuronales Recurrentes (RNR),
    Redes de Base Radial (RBR) \cite{royo2021}. Pero para este experimento se va a utilizar RNAs que es el algoritmo que 
    por el momento nos beneficiar\'a, cabe mencionar que no se usar\'an algoritmos genéticos, ya que si se implementa, se estará
    optimizando y el objetivo principal es utilizar el aprendizaje incremental para que acepte más datos de entrenamiento.
    \section{Consecuencias}

    Si el experimento funciona a la perfecci\'on ocurrir\'a lo siguiente:
    \begin{enumerate}
        \item Habrá menos olvido.
        \item El aprendizaje tomará menos tiempo.
    \end{enumerate}

    Esto sucederá porque se van a poder ingresar m\'as datos a nuestro almacén (REVISEN ESTAS PALABRAS, ALMACEN NO 
    SE HA UTILIZDO ANTES, LO PUEDEN CAMBIAR A PALABRAS MAS APROPIADAS, Y COMO MENCIONA ANTES) sin 
    tener que volver a codificar nuestro modelo.\\
    Los modelos de predicci\'on van a ser m\'as precisos, porque se podrán ingresar 
    datos mensualmente o hasta semanalmente, esto provocar\'a que el proyecto est\'e trabajando 
    con datos actuales.

    Pueden funcionar para proyectos y predicciones tan simples tanto la preiddci\'on climatol\'ogica
    hasta predicciones de la bolsa de valores y predicci\'on del Bitcoin.

    \section{Marco Teorico}

    \subsection{Revisión de la literatura}
        El humano tiene una forma de aprendizaje muy particular, la cual se basa del estudio, donde lee, escribe y practica acerca de
        su tema de interes, pero dicho aprendizaje se puede ir olvidando, esto es una acción muy común que a cualquier persona.
        Existen estudios donde se comenta que existen tres motivos del porque se olvidan las cosas, proviene parte de la regularización de las emociones,
        el como se adquirierón los conocimientos y porque el olvido es un proceso por el cual el ser humano transita a lo largo de su vida \cite{Nrby2015}. Pero cabe
        mencionar que esto no es lo único que causa la perdida de memoria, ya que existe la déficits de memoria. 

    \subsection{Aprendizaje Humano}
        Al momento de hablar del aprendizaje humano, se debe de hablar de la ciencia cognitiva, que es quien se encarga de descubrir esta incognita,
        esta cienca lo estudia de un modo multidisiplinario, el cual abarca las \'areas de: 
        \begin{itemize}
            \item La antropología.
            \item La lingüística.
            \item La filosofía.
            \item La sicología del desarrollo.
            \item La ciencia de la computación. 
            \item La neurociencia.
        \end{itemize}
        Con el metodo de esta cienca podemos descubrir dos tipos de aprendizaje que son:
        \begin{enumerate}
            \item Aprendizaje con Compresi\'on.
            \item Aprendizaje Activo.
        \end{enumerate}
    \subsection{Aprendizaje Incremental}
        Con el pasar de los años la tecnología a evolucionado, eso quiere decir que el Aprendizaje Automático se ha actualizado, que la 
        cantidad de datos va aumentado con más frecuencia y a los datos no se les da tanta importacia, es en este momento donde se implementa 
        este tipo de algoritmo,  

    \section{Metodología}
    El primer paso a realizar en esta investigación es recrear el código de John A. Bullinaria,
    para esto se va a utilizar el repositorio de Optical Digits, al momento de obtener estos datos
    ya se va a poder comenzar con la programación en Python.

    Al momento de comprobar que el código funciona como el algoritmo de John, se realizarán modificaciones
    que permitirán repartir las tazas de aprendizaje con estos valores.

    Para obtenerlo se usarán metodologías como el Backpropagation y funciones de activación tal como la Sigmoidal.

    Cuando los dos proyectos se tengan, se realizar\'a una comparación, donde se vera cual de estos dos experimentos
    es más eficaz en proyectos de la vida real.
    
\section{Cronograma de Actividades}

    \begin{figure}[H]
        \centering
        \includegraphics[width=\columnwidth]{diagramaGantt.png}
        \caption{Diagrama de Gantt}
        \label{fig:fig3}
    \end{figure}

\section{Organización del Capitulado}
	En el capitulo 2 se ver\'a lo que es el aprendizaje humano y el aprendizaje incremental con sus algoritmos, se describirán las redes neuronales artificiales.

En el capitulo 3 se implementar\'a el articulo de John A. Bullinaria, como funciona, resultado que da al pasar los datos que dice para comprobar que funciona como menciona en su art\'iculo. En el capitulo 4 se explicar\'a como se hará la modificación a su algoritmo, cuantas capas se van a poner, como se van a repartir las tazas de aprendizaje.

Posteriormente en el capitulo 5 se mostrar\'a una comparación de los resultados de ambos trabajos. En el capitulo 6 se verán las conclusiones y trabajo futuro.

    
\section{Organización del Capitulado}


En el capitulo 2 se ver\'a lo que es el aprendizaje humano y el aprendizaje incremental con sus algoritmos, se describirán las redes neuronales artificiales.

En el capitulo 3 se implementar\'a el articulo de John A. Bullinaria, como funciona, resultado que da al pasar los datos que dice para comprobar que funciona como menciona en su art\'iculo. En el capitulo 4 se explicar\'a como se hará la modificación a su algoritmo, cuantas capas se van a poner, como se van a repartir las tazas de aprendizaje.

Posteriormente en el capitulo 5 se mostrar\'a una comparación de los resultados de ambos trabajos. En el capitulo 6 se verán las conclusiones y trabajo futuro.
    
    %\printbibliography  

    %\bibliographystyle{acm}
    \bibliography{dataset}
    \bibliographystyle{plain}
    %\bibliography{BaseDatos2}

\end{document}