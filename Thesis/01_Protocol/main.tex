\documentclass[10pt,a4paper]{article}
\usepackage[utf8]{inputenc}
\usepackage[spanish,mexico]{babel} 
\usepackage{amsfonts}
\usepackage{amsmath}
\usepackage{amssymb}
\usepackage{multicol}
\usepackage{graphicx}
\usepackage{subcaption}
\usepackage{float}
\usepackage{lipsum}
\usepackage{xcolor}
\usepackage[rightcaption]{sidecap}
\usepackage{array}
\usepackage{framed}
\usepackage{color}
\usepackage{wrapfig}
\definecolor{shadecolor}{RGB}{224,238,238}
\usepackage{listings}
\usepackage[hidelinks]{hyperref}
\usepackage{pdfpages}
\usepackage{pdflscape}
\usepackage{verbatim}
\pagestyle{empty}
\usepackage[nottoc]{tocbibind}
\usepackage{times}
%\usepackage{apacite}

\graphicspath{{../05_Images/}}

% Márgenes
\usepackage[top=2.5cm, bottom=2.5cm, left=3.0cm, right=2.5cm]{geometry}

\title{Desarrollo de un Modelo de Aprendizaje Incremental de Redes Nueronales Artificiales Aplicado al Conjunto de
Datos Optdigit}

\makeindex

\begin{document}

    \includepdf{../03_Chapters/portada}
    \tableofcontents
    \listoffigures
    \maketitle
    \begin{abstract}
        El aprendizaje incremental es un área de la Inteligencia Artificial la cual permite agregar nuevo conocimiento 
        a un modelo (e.g. Redes Neuronales Artificiales) sin la necesidad de entrenar el modelo con toda la información 
        histórica de la tarea en cuestión \cite{bullinaria2009}. En el presente trabajo de investigación se ocupará el modelo de Redes 
        Neuronales Artificiales enfocada en la clasificación de dígitos escritos a mano usando el algoritmo de 
        entrenamiento de backpropagation, con redes Multi capa Perceptron y duplicación de pesos múltiples 
        simulando memoria a corto y largo plazo para mejorar los resultados presentados en \cite{bullinaria2009}.\\

        \textbf{Palabras claves:} Aprendizaje Incremental, Redes Multi Capa, Redes Neuronales Artificiales
    \end{abstract}

    \chapter{Introducción}
  La inteligencia artificial (IA) es una área del conocimiento que se enfoca en poder hacer máquinas que se enfocan en comportamiento y razonamiento humano, para que en momento dado, se pueda interactuar con una máquina. Así mismo, también es posible pensar que mucho del desarrollo en el área de inteligencia artificial, es el poder tener mejores herramientas que ayuden a las actividades diarias.\\

  En este sentido, un área de la IA es el llamado Aprendizaje Máquina, donde se estudian algoritmos que permiten aprender de forma auomática una tarea.
  Así, una de las técnicas más conocidas en la actualidad, dentro del área de IA son las Redes Neuronales Artificiales (RNAs), siendo técnicas que realizan procesos matemáticos para poder aprenderse tareas a resolver. Algunas áreas en las que son útiles las RNAs son en el aprendizaje de tareas no lineales, como la predicción de la capacidad de la red 5G, basada en el tráfico diario de este \cite{zhao2022} o clasificación, por ejemplo la clasificación de metales y rocas por medio de RNAs y lógica difusa \cite{salazar2013}. \\

  Las RNAs están formadas por neuronas artificiales que simulan a las biológicas. Así los procesos químicos que suceden en el cerebro, se simulan computacionalmente a través de señales que viajen a través de las neuronas artificiales, de aquí en adelante simplemente se referirá a ellas como "neuronas". Las neuronas en una RNA cuentan con una estructura distribuida en paralelo, presentando una buena habilidad de aprendizaje \cite{liu2015}.\\

  Dentro del aprendizaje máquina cuando una técnica, por ejemplo RNA, se enfoca en aprender una tarea, s conoce como \textit{algoritmo lineal sin memoria}, siendo uno de los métodos más empleados desde el inicio de las RNAs \cite{GiraudCarrier2000}. Sin embargo, si es necesario incorporar nueva información del problema,  es necesario volver a entrenar todo el modelo, considerando toda la información existente, esto es, la anterior y la nueva que acaba de llegar, es ahí donde nace el concepto de Aprendizaje Incremental, siendo un área enfocada en poder incorporar información del problema en cuestión, sin tener que volver a re-entrenar todo el modelo.\\

  Derivado del aprendizaje incremental se desprende el concepto de memoria dentro de la IA, analizando como un algoritmo de aprendizaje máquina puede olvidar la información que se us\'o en un entrenamiento previo al entrenar con información más reciente. Si se hace la analogía con los humanos, la memoria es un factor importante para estudiar considerando la perdida de información aprendida, así, este es un problema biol\'ogico, el cual tanto afecta a los humanos como a las m\'aquinas. Por ello, se han elaborado distintos experimentos para poder combatir esta problemática. Uno de estos es el caso de \cite{bullinaria2009}, el cual propone el manejo de RNAs con pesos dobles, donde la primer capa de pesos esta enfocada a comportarse como memoria a corto plazo, y la segunda como memoria a largo plazo.  Los experimentos mostrados en \cite{bullinaria2009} permiten notar un mejora en tareas de aprendizaje incremental, teniendo menos p\'erdida de información en comparación de implementaciones anteriores como el algoritmo  Learn++ \cite{li2008, Elwell2011}.\\ anteriores como el algoritmo  Learn++ \cite{li2008, Elwell2011}.\\
    
  Así, el presente trabajo de investigación esta enfocado en poder explorar nuevas configuraciones      Así, el presente trabajo de investigación esta enfocado en poder explorar nuevas configuraciones de pesos duplicados para poder extender el trabajo previamente presentado en \cite{bullinaria2009}.   de pesos duplicados para poder extender el trabajo previamente presentado en \cite{bullinaria2009}.
    \section{Planeamiento del Problema}

Las Redes Neuronales Artificiales (RNAs), en especial las \textit{Multilayer Perceptrons} (MLP), tienen la capacidad de aprender tareas y resolver problemas de predicción y clasificación. Utilizando algoritmos de aprendizaje como el \textit{Backpropagation}, es posible ajustar los pesos de una RNA para que pueda abordar una tarea específica. Las MLP son una forma común de RNA, compuestas por múltiples capas de neuronas, donde la capa de entrada recibe los datos y la capa de salida proporciona las predicciones. Las capas intermedias, también conocidas como capas ocultas, realizan transformaciones no lineales que permiten a la red aprender representaciones complejas de los datos. Sin embargo, muchas de las tareas que se resuelven en la vida cotidiana generan información adicional con el tiempo. Un ejemplo de esto es el comportamiento de una serie financiera o la predicción del clima en una determinada región. En este contexto, surge el aprendizaje incremental, un enfoque poco explorado que permite que el modelo aprenda nueva información sin la necesidad de reentrenar todo el modelo con los datos antiguos y nuevos.

En los modelos actuales de aprendizaje automático, cuando se utiliza un conjunto de datos para entrenar un modelo específico, dicho modelo es funcional solo para ese conjunto y la información que representa. Sin embargo, al ser necesario incorporar nueva información al modelo, se debe recolectar esa nueva información, añadirla a la ya existente y luego reentrenar todo el modelo para integrar los datos nuevos.

En este contexto, si una red entrenada con un primer conjunto de datos \(d_1\) debe entrenarse con un segundo conjunto de datos \(d_2\), al hacerlo, se perderá el conocimiento adquirido por \(d_1\). Si no se utiliza un modelo de aprendizaje incremental, se debe combinar \(d_1\) y \(d_2\) en un solo conjunto y reentrenar la RNA para incorporar el nuevo conocimiento (\(d_2\)). En cambio, si se emplea el aprendizaje incremental, la RNA se entrena inicialmente con \(d_1\) y luego con \(d_2\), manteniendo una mínima pérdida de información de \(d_1\). Si en el futuro se obtiene más información del problema (\(d_3\)), se entrenará la RNA con \(d_3\) usando el modelo incremental, minimizando la pérdida de datos de \(d_1\) y \(d_2\).

Este trabajo se basa en la investigación de \cite{bullinaria2009}, que utiliza una configuración de pesos duplicados en la RNA. En esta configuración, los pesos de la RNA se duplican y se asignan diferentes tasas de aprendizaje a las capas. La primera capa, asociada a una tasa de aprendizaje alta, simula la memoria a corto plazo al aprender rápidamente una nueva tarea y olvidar más rápido la información anterior. La segunda capa, con una tasa de aprendizaje baja, simula la memoria a largo plazo, aprendiendo más lentamente y olvidando menos la información previa. Al integrar ambas capas, la RNA pondera la salida para considerar tanto la nueva información adquirida como la olvidada en ambas capas de pesos.

El problema principal del aprendizaje incremental en \cite{bullinaria2009} es que, a medida que se incorporan nuevos conjuntos de datos, se pierde progresivamente el conocimiento adquirido de los primeros conjuntos, lo cual se vuelve menos útil si en el futuro se tienen 10 o 20 etapas de entrenamiento incremental con nuevos datos.

Por lo tanto, es crucial explorar nuevas configuraciones de RNAs que mejoren los métodos actuales para reducir la cantidad de información olvidada a medida que llega nueva información. Al igual que en investigaciones anteriores, este trabajo se basará en los conceptos de memoria a corto y largo plazo. Sin embargo, en lugar de duplicar los pesos y tener dos tasas de aprendizaje, se propondrá la idea de crear más copias de los pesos, cada una con una tasa de aprendizaje distinta.

    \section{Objetivos}
    Diseñar una red neuronal artificial para aprendizaje incremental basada en el principio de la memoria a corto y largo plazo, buscando usar más de dos capas de pesos duplicados para el reconocimiento de dígitos, y con una menor perdida de información que trabajos previos.
    \subsection{Objetivos Particulares}
        \begin{enumerate}
            \item Implementar el algoritmo mostrado en \cite{bullinaria2009} para el reconocimiento de dígitos con aprendizaje a corto y largo plazo con los parámetros que ahí se indican.
            \item Obtener el conjunto de datos de Optical Digits, limpiar los datos y prepararlos segun lo indicado con \cite{bullinaria2009}.
            \item Separar el conjunto de entrenamiento y de prueba de acuerdo a lo que se explica en el artículo de Bullinaria yprobar el primer código implementado en miras de comprobar  los resultados previamente mostrados en \cite{bullinaria2009}.
            \item Tomando como base el algoritmo implementado,  y extenderlo para permitir mas de dos pesos duplicados, aplicando el conjunto de datos previamente mostrado.
            \item Comparar ambas implementaciones en búsca de una reducción significativa de las tasas de aprendizaje con respecto a trabajos previos en la literatura.
        \end{enumerate}
    \section{Hipotesis}
    Al tener m\'as conexiones en un modelo de red neuronal usualmente se tendr\'a mejor ajuste de los 
    problemas, ya que al momento de predecir este ser\'a mucho m\'as eficiente de manejar.
    \section{Justificación}

Las redes neuronales artificiales permiten el aprendizaje automático y la resolución de distintos problemas. Sin embargo, como se mencionó anteriormente, las técnicas de aprendizaje automático presentan una deficiencia importante: cuando se incorporan nuevos bloques de datos, se observa un deterioro en el rendimiento del aprendizaje y un olvido de la información previamente aprendida \cite{bullinaria2009}. 

Aunque los resultados obtenidos hasta ahora no han sido perfectos, la memoria a corto plazo de las redes neuronales tiende a olvidar con el tiempo, lo cual es una limitación. En contraste, los seres humanos pueden aprender nuevas tareas o información de un problema sin olvidar de manera significativa lo que previamente aprendieron. Aunque biológicamente los humanos tienen una estructura cerebral que les permite aprender y retener mejor la información, actualmente no existe ningún procedimiento que permita modificar la estructura del cerebro para mejorar la retención y reducir el olvido.

Desde un punto de vista computacional, no hay limitaciones inherentes que impidan experimentar con nuevas configuraciones de redes neuronales artificiales (RNAs) para lograr que toda la información ingresada en el modelo se acumule, sin problemas de almacenamiento, y sin olvidar la información previa. Esto podría ser beneficioso en diversas aplicaciones.

En un escenario en el que no se utilice aprendizaje incremental, la llegada de nueva información implicaría la necesidad de volver a entrenar todo el modelo con la información anterior y la nueva. Esto sería un proceso costoso en términos de cómputo y energía, ya que el entrenamiento de una RNA es uno de los cuellos de botella principales, lo que significa un alto consumo de recursos. En cambio, si solo se entrena con la nueva información, se podrían reducir tanto el tiempo como el consumo energético.

Existen varias herramientas que permiten codificar redes neuronales artificiales utilizando librerías preexistentes. En esta investigación, se destacan dos plataformas principales: Microsoft Azure y Google Colab. Azure ofrece una máquina virtual para programar en Python, mientras que Google Colab proporciona un entorno similar, pero de forma gratuita. Ambas herramientas permiten trabajar con Python, un lenguaje de programación libre y fácil de depurar, ideal para el desarrollo de aplicaciones de Machine Learning.

Una de las librerías más populares para Machine Learning es TensorFlow, que facilita la creación y entrenamiento de redes neuronales, permitiendo detectar patrones y razonamientos. Debido a sus capacidades y a su compatibilidad con Deep Learning, TensorFlow será la librería utilizada en esta investigación, particularmente porque es compatible con Keras, un framework de alto nivel que se ejecuta sobre TensorFlow. Keras simplifica los procesos de experimentación rápida y es ideal para su ejecución en plataformas como Google Colab.


    \chapter{Delimitación}
\label{sec:delimitation}
	
	
    En la presente investigación solamente se utilizará redes neuronales del tipo perceptr\'on multicapa,  donde 
    cabe mencionar que este no es el único tipo de red que existe, i.e., también se tiene Redes Neuronales Convolucionales (RNC), Redes Neuronales Recurrentes (RNR) o bien Redes de Base Radial (RBR) \cite{royo2021}. Así mismo, no se abordará el uso de técnicas d eoptimiación como lo son los algoritmos genéticos, y unicamente se limitará a explorar la mejora en rendimiento al tener más de dos capas duplocadas de pesos en la red con aprendizaje incremental.  Así mismo, no se abordarán modelos como las redes profundas u otro conjunto de datos y se limitará el trabajo a lo antes mencionado.



    \section{Consecuencias}

Si el experimento tiene éxito, se espera que ocurran las siguientes consecuencias:
\begin{enumerate}
    \item Se reducirá el olvido de la información previamente aprendida.
    \item El tiempo de aprendizaje será menor.
\end{enumerate}

Como se mencionó previamente, será posible incorporar nuevos datos sin necesidad de reentrenar el modelo con toda la información existente. \\

Las tareas de clasificación o predicción podrán integrar información nueva en las redes neuronales artificiales sin la necesidad de volver a entrenar el modelo con todos los datos históricos, lo que permitirá reducir el cuello de botella asociado al proceso de entrenamiento.

    \section{Marco Teorico}

    \subsection{Revisión de la literatura}
        El humano tiene una forma de aprendizaje muy particular, la cual se basa del estudio, donde lee, escribe y practica acerca de
        su tema de interes, pero dicho aprendizaje se puede ir olvidando, esto es una acción muy común que a cualquier persona.
        Existen estudios donde se comenta que existen tres motivos del porque se olvidan las cosas, proviene parte de la regularización de las emociones,
        el como se adquirierón los conocimientos y porque el olvido es un proceso por el cual el ser humano transita a lo largo de su vida \cite{Nrby2015}. Pero cabe
        mencionar que esto no es lo único que causa la perdida de memoria, ya que existe la déficits de memoria. 

    \subsection{Aprendizaje Humano}
        Al momento de hablar del aprendizaje humano, se debe de hablar de la ciencia cognitiva, que es quien se encarga de descubrir esta incognita,
        esta cienca lo estudia de un modo multidisiplinario, el cual abarca las \'areas de: 
        \begin{itemize}
            \item La antropología.
            \item La lingüística.
            \item La filosofía.
            \item La sicología del desarrollo.
            \item La ciencia de la computación. 
            \item La neurociencia.
        \end{itemize}
        Con el metodo de esta cienca podemos descubrir dos tipos de aprendizaje que son:
        \begin{enumerate}
            \item Aprendizaje con Compresi\'on.
            \item Aprendizaje Activo.
        \end{enumerate}
    \subsection{Aprendizaje Incremental}
        Con el pasar de los años la tecnología a evolucionado, eso quiere decir que el Aprendizaje Automático se ha actualizado, que la 
        cantidad de datos va aumentado con más frecuencia y a los datos no se les da tanta importacia, es en este momento donde se implementa 
        este tipo de algoritmo,  

    \chapter{Metodología}
	
	
    El primer paso a realizar, es recrear el código mostrado en \cite{bullinaria2009}, el describe la implementación de una red neuronal multicapa usando el algoritmo de entrenamiento backpropagation en el lenguaje de programación Python. 
    Así mismo, se utilizará el conjunto de datos de Optical Digits, en donde se tendrá que preprocesar los datos, para eliminar registros inválidos.\\
    
    Posteriormente, se implementará una extensión del código, donde se experimentará con mas de dos capas de pesos duplicados para mejorar la tasa de olvido de información al momento de usar el aprendizaje incremental.  Para ello se explorará incrementando gradualmente el número de capas de pesos duplicados.\\
    
    Finalmente, cuando los resultados se obtengan se realizará una comparación, de los resultado del algoritmo base con los resultados del algoritmo extendido.
    



    
\section{Organización del Capitulado}


En el capitulo 2 se ver\'a lo que es el aprendizaje humano y el aprendizaje incremental con sus algoritmos, se describirán las redes neuronales artificiales.

En el capitulo 3 se implementar\'a el articulo de John A. Bullinaria, como funciona, resultado que da al pasar los datos que dice para comprobar que funciona como menciona en su art\'iculo. En el capitulo 4 se explicar\'a como se hará la modificación a su algoritmo, cuantas capas se van a poner, como se van a repartir las tazas de aprendizaje.

Posteriormente en el capitulo 5 se mostrar\'a una comparación de los resultados de ambos trabajos. En el capitulo 6 se verán las conclusiones y trabajo futuro.
    
    %\printbibliography  
    %\bibliographystyle{acm}
    %\bibliographystyle{plain}
    \bibliography{../02_References/dataset}
    %\bibliography{BaseDatos2}

\end{document}