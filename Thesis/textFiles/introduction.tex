\chapter{Introducción}
  La inteligencia artificial (IA) es una área del conocimiento que se enfoca en poder hacer máquinas que se enfocan en comportamiento y razonamiento humano, para que en momento dado, se pueda interactuar con una máquina. Así mismo, también es posible pensar que mucho del desarrollo en el área de inteligencia artificial, es el poder tener mejores herramientas que ayuden a las actividades diarias.\\

  En este sentido, un área de la IA es el llamado Aprendizaje Máquina, donde se estudian algoritmos que permiten aprender de forma auomática una tarea.
  Así, una de las técnicas más conocidas en la actualidad, dentro del área de IA son las Redes Neuronales Artificiales (RNAs), siendo técnicas que realizan procesos matemáticos para poder aprenderse tareas a resolver. Algunas áreas en las que son útiles las RNAs son en el aprendizaje de tareas no lineales, como la predicción de la capacidad de la red 5G, basada en el tráfico diario de este \cite{zhao2022} o clasificación, por ejemplo la clasificación de metales y rocas por medio de RNAs y lógica difusa \cite{salazar2013}. \\






