\section{Metodología}

Este trabajo se divide en tres etapas principales: la recreación del modelo base, la implementación de una extensión al modelo, y la comparación de los resultados obtenidos. A continuación, se describen cada una de estas etapas en detalle:

\begin{enumerate}
    \item \textbf{Recreación del código base} \\
    Como primer paso, se recreará el código presentado en \cite{bullinaria2009}, que describe la implementación de una red neuronal multicapa utilizando el algoritmo de entrenamiento \textit{backpropagation} en el lenguaje de programación Python. Este código servirá como base para las siguientes etapas. \\
    Además, se empleará el conjunto de datos \textit{Optical Digits}, el cual será sometido a un proceso de preprocesamiento para eliminar registros inválidos y garantizar la calidad de los datos utilizados en el entrenamiento y validación del modelo.

    \item \textbf{Extensión del modelo base} \\
    Posteriormente, se desarrollará una extensión al modelo base, con el objetivo de experimentar con configuraciones más complejas de la red neuronal. Específicamente, se agregarán más de dos capas de pesos duplicados, lo cual se espera que mejore la retención de información al emplear el aprendizaje incremental. \\
    Para este propósito, se realizarán experimentos incrementando gradualmente el número de capas de pesos duplicados, con el fin de analizar el impacto de esta configuración en la capacidad de aprendizaje de la red.

    \item \textbf{Comparación y análisis de resultados} \\
    Una vez obtenidos los resultados de las simulaciones, se realizará una comparación entre el rendimiento del modelo base y el modelo extendido. Este análisis incluirá métricas clave, como la precisión del modelo, la tasa de olvido de información y el desempeño en el aprendizaje incremental. \\
    Los resultados permitirán evaluar si las modificaciones introducidas en la arquitectura de la red neuronal mejoran significativamente su desempeño en comparación con la implementación original.
\end{enumerate}



