
\section{Organización del Capitulado}


		En el capitulo 2 se explicará lo que son y como funcionan las redes neuronales artificiales, así como la función de activación que es un método que utilizan estas.
		Se describirán los tipos de redes neuronales, tanto de perceptron multicapa como convolucionales, y el algoritmo backpropagation.
		Se mencionará el conjunto de datos Optdigit que se utilizará para el entrenamiento y prueba de los algoritmos.
		Y para terminar se describirá el aprendizaje incremental y su algoritmo.\\
		
		En el capitulo 3 se implementará el algoritmo de John A. Bullinaria, se verificará su funcionamiento y los resultados que da al pasar los datos que dice para comprobar que es como menciona en su artículo.\\
		En el capitulo 4 se explicará como se se extendió el algoritmo base permitiendo el uso de mas de dos pesos duplicados y se aplicaran los mismos datos de entrenamiento y de prueba que al algoritmo base.\\
		
		Posteriormente, en el capitulo 5 con los resultados obtenidos, se mostrará una comparación de los resultados de ambos trabajos para notar si hubo una reducción significativa en las tasas de aprendizaje. En el capitulo 6 se verán las conclusiones y trabajo futuro.