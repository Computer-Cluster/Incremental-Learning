\documentclass[letter, 12pt, twoside]{report}  
\usepackage[spanish,mexico]{babel} 
\usepackage[utf8]{inputenc}
\usepackage{setspace}
\usepackage[top=2.4cm, bottom=2.7cm, left=3.3cm, right=2.0cm]{geometry}   % with the script is lower the text, with kile it put +o- 1 cm above, check when pinted in the school
%\usepackage[square, comma, sort&compress]{natbib}
\usepackage{makeidx}
\setcounter{secnumdepth}{3}
\setcounter{tocdepth}{3} 		
\usepackage{url}
\usepackage{subfig}  % para usar la función subfigura
\usepackage{epsfig}  % para trabajar con figs eps 
\usepackage{multirow}
\usepackage{bigstrut}
\usepackage{booktabs}
\usepackage{rotating} 				% for vertical text in tables with  \begin{sideways}Paper\end{sideways} &\begin{sideways}Static\end{sideways} \\
\usepackage{algorithm}
%\usepackage[numbered]{algo}
%\usepackage[printonlyused,withpage]{acronym}
\usepackage{hyphenat} 			%hyphenation of compound words
\usepackage{array} 			% to increse space between rows in tables using \setlength{\extrarowheight}{1.5pt}
%\usepackage{multicol}
\usepackage{pdflscape} 			% for landscape some pages
\usepackage{textcomp} 			% for the +- sign \textpm
\usepackage{graphicx}

\graphicspath{ {images/} }

\makeindex

\begin{document}

    \begin{titlepage}
    \begin{center}
    \vspace*{1.0in}
    {\LARGE Aprendizaje Incremental para la Tarea de Reconocimiento de Digitos con Redes Neuronales Artificiales}
    \par
    %\vspace{0.2in}
    %{\LARGE hh}
    %\par
    \vspace{0.5in}
    {by}
    \par
    \vspace{0.2in}
    {\large Gonz\'alez Hern\'andez Luis \'Angel}
    \par					% blank line
    %\vfill 					%strech vertical space so that it fills all empty space 
    \vspace{2in}
    
    \par
    \vspace{2in}
    \begin{flushright}
     
    
    CUUAEM VM \\
    UAEM\
    Diciembre 2022
    
    \end{flushright}
    
    \end{center}
\end{titlepage}
    \pagenumbering{roman}
    \newpage
    \thispagestyle{empty}
    % Empty page


    % put double spacing
    \doublespacing

    \thispagestyle{empty}
    %1-2 pages
    \begin{abstract}

        El aprendizaje incremental es un área de la Inteligencia Artificial la cual permite agregar nuevo conocimiento 
        a un modelo (e.g. Redes Neuronales Artificiales) sin la necesidad de entrenar el modelo con toda la información 
        histórica de la tarea en cuestión %\cite{bullinaria2009}. En el presente trabajo de investigación se ocupará el modelo de Redes 
        Neuronales Artificiales enfocada en la clasificación de dígitos escritos a mano usando el algoritmo de 
        entrenamiento de backpropagation, con redes Multi capa Perceptron y duplicación de pesos múltiples 
        simulando memoria a corto y largo plazo para mejorar los resultados presentados en %\cite{bullinaria2009}.\\

    \end{abstract}

    \newpage
    \thispagestyle{empty}
    % Empty page

    % dedication
    \newpage
    \thispagestyle{empty}

    %  \addvspace{3in}
    \vspace*{2in}


    \noindent \hspace{2in} Dedicatoria 

    \newpage
    \thispagestyle{empty}
    % Empty page
    ~

    \chapter*{Agradecimientos}
    \thispagestyle{empty}


    Sus agradecimientos

    \tableofcontents

    \listoffigures

    \listoftables

    % \listofalgorithms


    \newpage
    \thispagestyle{empty}
    % Empty page
    ~

    % A c r o n y m s
    \chapter*{Lista de Acronimos}


%%%%%%%%%%%%%%%%%%%%%%%%%%%%%%%%%%%%%%%%%%%%%%%%%%%%%%%%%%%%%%%%%%%%%%%%%%%%%%%%%%%%%%%%%%%%%%%%%%%%%%%%%%%%%%%%%%%%%%%%%%%%%%%%%
% Rules for this pacage
% \ac{ acronym } 	put it, put full acronym if it is the first time
% \acresetall 		flushed the memory of macro \ac, so it will print the full acroym the first time any of them is called
% \acf{} 		Full name is printed, and the acronym is in brackets
% \acs{}		short version
% \acl{} 		expanded acronym without even mentioning the acronym

% \acp			Works in the same way as \ac, but makes the short and/or long forms into plurals.
% \acfp			Works in the same way as \acf, but makes the short and long forms into plurals.
% \acsp			Works in the same way as \acs, but makes the short form into a plural.
% \aclp			Works in the same way as \acl, but makes the long form into a plural.
% \acfi			Prints the Full Name acronym (\acl) in italics and the abbreviated form (\acs) in upshaped form.
% \acused		Marks an acronym as used, as if it had been called with \ac, but without
% 			printing anything. This means that in the future only the short form of the
% 			acronym will be printed.
% 
% \acsu			Prints the short form of the acronym and marks it as used.
% \aclu			Prints the long form of the acronym and marks it as used. Example: \acl{lox}/\acl{lh2} (\acsu{lox}/\acsu{lh2})
% \...*			The following commands do the same as their unstarred forms, except that the
% 			acronym will not be marked as used. If you work with the ’onlyused’ option then
% 			macros which have only been used with starred commands will not show up.
% 			\ac*, \acs*, \acl*, \acf*, \acp*, \acsp*, \aclp*, \acfp*, \acfi*, \acsu* and
% 			\aclu*.
%
% For more information see the manual: /media/dat/LIBROS/Latex/acronyms/acronym.pdf
%%%%%%%%%%%%%%%%%%%%%%%%%%%%%%%%%%%%%%%%%%%%%%%%%%%%%%%%%%%%%%%%%%%%%%%%%%%%%%%%%%%%%%%%%%%%%%%%%%%%%%%%%%%%%%%%%%%%%%%%%%%%%%%%%




% Previous version

%\begin{table}[b]
  	%\caption{Multiple-step, closed-loop or iterate forecasting}
%		\centering
%			\begin{tabular}{llll}
				%\hline\noalign{\smallskip}
				%${Forecast}$ & $Inputs$ \\
				%\noalign{\smallskip}
				%\hline
				%\noalign{\smallskip}
%& MSP & & Multi step prediction \\
%& SSP & & Single step prediciton \\ 
				
				%\hline
%			\end{tabular}\\
%			\vspace{-2mm}
%		\label{MultipleForecasting}
%	\end{table}

    %\newpage

    \pagenumbering{arabic}

    % put double spacing
    \doublespacing



    % Each one is a chapter 
    \section{Introduccion}
Se requiere el uso de las redes neuronales para poder dar predicciones de sucesos los cuales pasan en nuestras vidas, al momento de decir predicción lo primero que se 
nos viene a la mente es la astrología, pero no todo es astrología, nosotros tenemos el poder de realizar predicciones con el uso de tecnología. En Microsoft los del 
departamento de AI son conocidos como \textbf{AIWizzard}, ya que nosotros fabricamos magia, pero que pasa si implementamos esta magia en la vida de un estudiante o en la vida de 
un profesional,  claramente vamos a poder observar un cambio enorme en el cual su desempeño va a mejorar ya que conocerá, aprenderá, mejorará y optimizará todos sus labores. \\
En especial el uso de las redes neuronales es un campo importante en la AI el cual nos va a permitir mejoras gigantes en nuestro mundo.

El aprendizaje incrementado es un método el cual a sido implementado en el área de la inteligencia artificial, ya que al realizar tareas de inteligencia artificial especificas esta rama nos 
ayuda a optimizarla para que el algoritmo sea más eficiente

    \newpage
    \thispagestyle{empty}
    % Empty page
    
    \section{Marco Teorico}

    \subsection{Revisión de la literatura}
        El humano tiene una forma de aprendizaje muy particular, la cual se basa del estudio, donde lee, escribe y practica acerca de
        su tema de interes, pero dicho aprendizaje se puede ir olvidando, esto es una acción muy común que a cualquier persona.
        Existen estudios donde se comenta que existen tres motivos del porque se olvidan las cosas, proviene parte de la regularización de las emociones,
        el como se adquirierón los conocimientos y porque el olvido es un proceso por el cual el ser humano transita a lo largo de su vida \cite{Nrby2015}. Pero cabe
        mencionar que esto no es lo único que causa la perdida de memoria, ya que existe la déficits de memoria. 

    \subsection{Aprendizaje Humano}
        Al momento de hablar del aprendizaje humano, se debe de hablar de la ciencia cognitiva, que es quien se encarga de descubrir esta incognita,
        esta cienca lo estudia de un modo multidisiplinario, el cual abarca las \'areas de: 
        \begin{itemize}
            \item La antropología.
            \item La lingüística.
            \item La filosofía.
            \item La sicología del desarrollo.
            \item La ciencia de la computación. 
            \item La neurociencia.
        \end{itemize}
        Con el metodo de esta cienca podemos descubrir dos tipos de aprendizaje que son:
        \begin{enumerate}
            \item Aprendizaje con Compresi\'on.
            \item Aprendizaje Activo.
        \end{enumerate}
    \subsection{Aprendizaje Incremental}
        Con el pasar de los años la tecnología a evolucionado, eso quiere decir que el Aprendizaje Automático se ha actualizado, que la 
        cantidad de datos va aumentado con más frecuencia y a los datos no se les da tanta importacia, es en este momento donde se implementa 
        este tipo de algoritmo,  

    \include{textFiles/prueba}


    %\bibliographystyle{acm}
    \bibliographystyle{ieeetr}%plain}
    \bibliography{bibliography}
    %\bibliography{BaseDatos2}


%\printindex
\end{document}

