\section{Planeamiento del Problema}
	
	
    Las Redes Neuronales Artificiales tienen la habilidad de poderse aprender una tarea 
    y poder hacer o resolver tareas de predicción o clasificación básicamente. En este 
    sentido, con algoritmos de aprendizaje como el Backpropagation, permite ajustar los 
    pesos de una RNA para que esta pueda empezar a resolver una tarea particular.  No 
    obstante, muchas de las tareas que se resuelven en la vida diaria, van generando mas 
    información con el tiempo, por ejemplo, el comportamiento de un una serie financiera, 
    o bien la predicción del clima en una determinada región. Así,  se tiene el aprendizaje 
    incremental, siendo un método poco explorado  enfocado en poder aprender nueva información del 
    problema, sin tener que volver a entrenar todo el modelo con la información anterior y la 
    nueva que acaba de llegar, esto es, en los modelos actuales de aprendizaje máquina, si se usa 
    un conjunto de datos para entrenar un modelo en especifico, dicho modelo es funcional para 
    dicho conjunto de datos y la información que ello representa. Sin embargo, si es necesario 
    incorporar nueva información al modelo, es necesario recolectar dicha información nueva, 
    agregarla a la que se tenia anteriormente, y volver a entrenar todo el modelo para que \'este 
    pueda incorporar la nueva información.\\

    De esta forma, una red ya entrenada con un primer conjunto de datos $d_{1}$ se desea entrenar 
    con un nuevo conjunto de datos del mismo problema $d_{2}$, al entrenar la red con $d_{2}$ se 
    perderá en conocimiento aprendido por $d_{1}$.  Si no se desea utilizar un modelo de aprendizaje 
    incremental, se tendría que juntar el conjunto $d_{1}$ y $d_{2}$ en un solo conjunto y volver a 
    entrenar la RNA para así poder incorporar el nuevo conocimiento ($d_{2}$) a la RNA. Si se desea 
    utilizar el método de aprendizaje incremental, se puede entrenar en un primer momento a la red 
    con el conjunto $d_{1}$, posteriormente con $d_{2}$ teniendo poca perdida de información de $d_{1}$. 
    Si más adelante llega mas información del problema ($d_{3}$) que se desee incorporar a la base 
    de conocimientos de la RNA, entonces solo habrá que entrenar la RNA con $d_{3}$ usando el modelo 
    incremental para tener una p\'erdida mínima de información de $d_{1}$ y $d_{2}$. \\

    De esta forma, el presente trabajo tomar\'a como base la investigación de \cite{bullinaria2009}, 
    en donde utiliza una configuración de pesos dobles (a una RNA se duplican todos sus pesos), 
    donde a una capa de pesos duplicados se enlaza con una tasa de aprendizaje alta, para simular un 
    rápido aprendizaje, y por ende simular lo que sería memoria a corto plazo. En su contraparte, 
    la segunda capa de pesos duplicados, se enlaza con una tasa de aprendizaje baja para aprender 
    lentamente un problema, simulando la memoria a largo plazo.  Es decir, al momento de aprender una 
    tarea nueva,  una capa de pesos aprenderá muy rápido la nueva tarea (tasa alta de aprendizaje) y 
    por ende olvidará más rápidamente los datos ya aprendidos anteriormente, por otro lado, la segunda 
    capa de pesos duplicados, aprenderá muy lentamente los nuevos datos que llegan, y por ende olvidando 
    poco la información que anteriormente se aprendió.  Considerando ello y también que la 
    RNA trabaja en conjunto con ambas capas de pesos duplicados, se pondera la salida de la RNA para 
    que pueda contemplar la información nueva que se acaba de agregar a la RNA en ambas capas, así como 
    la información que se acaba de olvidar de ambas capas de pesos.\\

    El problema principal del aprendizaje incremental mostrado en \cite{bullinaria2009}, es que entre 
    más conjuntos de datos nuevos que lleguen, mas se olvidaran los primeros conjuntos que se aprendieron, 
    lo cual no es tan útil si se contempla que en el futuro de una RNA podrán existir 10 o 20 etapas de 
    entrenamiento incremental con nuevos conjuntos de datos que se vayan recolectando.\\

    Por ello es indispensable poder explorar nuevas configuraciones de RNAs que permitan mejorar los 
    métodos actuales para permitir una menor cantidad de olvido conforme llegue nueva información al 
    modelo. Donde al igual que en trabajos anteriores,  el presente trabajo se basará en conceptos de 
    memoria a corto y largo plazo, y en lugar de hacer una copia de los pesos actuales y tener dos 
    tasas de aprendizaje, una rápida para simular la memoria a corto plazo, y otra tasa de aprendizaje 
    para simular la memoria a largo plazo, se explorará por hacer mas copias de los pesos, teniendo más 
    tasas de aprendizaje que operen en cada una de dichas copias.
    