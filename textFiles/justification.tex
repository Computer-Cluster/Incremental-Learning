\section{Justificación}

    Las redes neuronales permiten el aprendizaje automático y la resolución 
    de distintos problemas,  pero como se comentó anteriormente,  las técnicas 
    de aprendizaje máquina, tienen una deficiencia que es al momento de aumentar 
    los nuevos bloques de datos que llegan para aprender,  se obtiene un deterioro 
    en el rendimiento de aprendizaje de información y olvido de la información anterior
    \cite{bullinaria2009}.   (REFIVEN COMO VOY PONEINDO LA REDACCIÓN DE LO QNTERIOR, Y 
    CO NELLO TRATEN DE CAMBIAR LO QUE VIENE PARA QUE LE MOJOREN EL ESTILO, DE AQUÍ EN 
    ADELANTE SOLO LES HARE OBSERVACIONES EN ESPERA QUE USTEDES MEJOREN EN GENERAR LA 
    REDACCIÓN)
    
    Los resultados que se han obtenido no funcionan a la perfección, la 
    memoria a corto plazo olvida poco pero va olvidando, y lo ideal sería que no 
    olvidara. Biológicamente los humanos pueden aprender nuevas tareas, o información 
    nueva de un problema, y no olvida de forma significativa lo que anteriormente 
    aprendió, no obstante eso no pasa actualmente con las RNA y en general con 
    cualquier algoritmo de aprendizaje máquina. En otro sentido, los humanos 
    ya tienen cierta configuración en el cerebro que les permite aprender como 
    se hace actualmente,  y se puede afirmar que por el momento no hay ningún 
    procedimiento (quirúrgico o no) que permita modificar la estructura del cerebro 
    para aprender mas y olvidar menos. 

    No obstante,  computacionalmente nada puede impedir que se experimente con más 
    configuraciones y llegar al punto en donde toda la información que ingrese a un 
    modelo (por ejemplo RNAs) se acumule, y si no hay problema de almacenamiento que se siga 
    acumulando y que no olvide, esto podría ser bueno en diversas situaciones.

    Desde el punto de vista computacional, si llega nueva información y no se ocupa aprendizaje 
    incremental, ello implicar\'a volver a entrenar todo el sistema con la información anterior y 
    la actual (por ejemplo $d_{1}$ y $d_{2}$) y considerando que una de las desventajas que tienen la 
    RNAs es que el entrenamiento es un cuello de botella, siendo este donde se lleva la mayor 
    parte de cómputo y por consiguiente de energía. Ello implica que volver a entrenar con toda 
    la información acumulada, gastará más energía y tiempo, que si solo se entrena con la nueva 
    información que llega al modelo. En su contraparte,  existe una gran variedad de herramientas 
    las cuales permiten codificar una red neuronal artificial con librerías ya preexistentes, por 
    el momento se expondrán solo 2 empresas, siendo estas las más importantes: Microsoft y Google. 
    La primera cuenta con la plataforma de Azure que renta una maquina virtual donde se puede 
    programar en Python.  Por el contrario, Google cuenta con Google Colab que igual brinda 
    una m\'aquina virtual para realizar experimentos de Maching Learning, la \'unica diferencia 
    contra Azure es que dicha herramienta es gratuita, una similitud que tienen es que en las 
    dos se puede programar en el mismo lenguaje.

    Como se observa, ambas herramientas permiten la programación en Pyhton y esto se debe a que 
    dicho lenguaje es una herramienta de software libre que no requiere licencia, es relativamente 
    fácil poder depurar un código y permite acelerar más el desarrollo de aplicaciones,  a diferencia 
    de otros lenguajes más estructurados como C o Java, adem\'as tiene m\'as librerías para el 
    desarrollo de Maching Learning, por ejemplo, TensorFlow, Numpi, entre otras. \\

    TensorFlow es una librería de Python que permite construir y entrenar redes neuronales para 
    detectar patrones y razonamientos usados por los humanos, en la presente investigaci\'on se 
    usar\'a dado a que favorece la creaci\'on de una RNA, permite la elaboraci\'on de cualquier 
    tipo de algoritmo de Machine Learning, cabe mencionar que también se puede usar para Deep Learning, 
    facilita la adquisici\'on de datos modelos de capacitaci\'on, predicciones y refinamiento de 
    resultados, est\'a disponible para el uso en computadores personales, pero es recomendado 
    usarlo en su propio editor en la nube que es Colab. \\ 

    Keras (MISMO COMENTARIO QUE LES PONGO DE TENSORFLOW)es un framework de alto nivel para 
    el aprendizaje, escrito en Python y capaz de correr sobre los frameworks TensorFlow. Fue 
    desarrollado con el objetivo de facilitar un proceso de experimentación rápida. Diseñado 
    para construir por bloques la arquitectura de cada red neuronal, incluyendo redes 
    convolucionales y modelos recurrentes, que son las que permiten, junto a los bloques 
    "más tradicionales", entrenar aprendizaje profundo.