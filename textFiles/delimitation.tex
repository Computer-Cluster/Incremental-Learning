\section{Delimitación}
\label{sec:delimitation}

    En la siguiente investigación solamente se van a utilizar redes neuronales artificiales, 
    cabe mencionar que este no es el único tipo de red, porque también se tiene lo que son Redes Neuronales Monocapa,
    Redes Neuronales Perceptrón Multicapa (RNPM), Redes Neuronales Convolucionales (RNC), Redes Neuronales Recurrentes (RNR),
    Redes de Base Radial (RBR) \cite{royo2021}. Pero para este experimento se van a utilizar RNAs que es el algoritmo que 
    por el momento beneficiar\'a a la presente investigación, cabe mencionar que no se usar\'an algoritmos genéticos, ya que si se implementan, se estará
    optimizando y el objetivo principal es utilizar el aprendizaje incremental para que acepte más datos de entrenamiento.
    
    (Esto es lo que lo que el puso en el marco teorico, y tambien deje su comentario ahi para que te ubiques va)
    Así como las redes convolucionales, también existen las redes profundas(deep learning), sin embargo, como se comenta en esta sección \label{sec:delimitation} el presente trabajo solo se basará en redes del tipo multicapa perceptron usando el algoritmo de bacpkpropagation, donde de ser demostrados los principios descritos anteriormente, se podrá aplicar a cualquier otro tipo de algoritmo de aprendizaje.