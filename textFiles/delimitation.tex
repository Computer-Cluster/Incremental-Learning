\section{Delimitación}
\label{sec:delimitation}

    En la siguiente investigación solamente se van a utilizar redes neuronales de perceptr\'on multicapa, 
    cabe mencionar que este no es el único tipo de red, porque también se tiene lo que son Redes Neuronales Monocapa,
    Redes Neuronales Perceptrón Multicapa (RNPM), Redes Neuronales Convolucionales (RNC), Redes Neuronales Recurrentes (RNR),
    Redes de Base Radial (RBR) \cite{royo2021}. Pero para este experimento se van a utilizar RNAs que es el algoritmo que 
    por el momento beneficiar\'a a la presente investigación, cabe mencionar que no se usar\'an algoritmos genéticos, 
    ya que si se implementan, se estará optimizando y el objetivo principal es utilizar el aprendizaje incremental 
    para que acepte más datos de entrenamiento.
    
    La inteligencia artificial se divide en dos, el Aprendizaje Maquina (Maching Learnin) y el Aprendizaje 
    Profundo (Deep Learning. El Maching Learning son algoritmos que permiten a una computadora realizar 
    procesos sin la intervensi\'on del ser humano y el Deep Learning es una parte del aprendizaje maquina
    que usa algoritmos de alto nivel, los cuales pretenden imitar una red neuronal humana. En el Maching Learning
    tambi\'en se emulan las RNAs, a excepci\'on de que en este tipo de inteligencia artificial no usa tantas
    capas ocultas, como el caso del aprendizaje profundo.
