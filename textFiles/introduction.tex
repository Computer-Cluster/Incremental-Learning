\section{Introduccion}

El aprendizaje incremental es un método el cual a sido implementado en el área de la inteligencia artificial, ya que al realizar tareas especificas de dicha rama nos 
ayuda a optimizarla para que el algoritmo sea más eficiente.
Cualquier tipo de aprendizaje puede ser considerado aprendizaje incremental si el problema a resolver tiene un entrenamiento simple, además este tipo de algoritmo es conocido como \textit{algoritmo lineal sin memoria},
en la mayor\'ia de los casos este tipo de aprendizaje es el preferido o favorito por los desarrolladores \cite{GiraudCarrier2000}.

Las redes neuronales artificiales (RNAs) son procesos matem\'aticos los cuales son utilizados en el \'area de Machine Learning para 
la resoluci\'on de problemas no lineales, estos deben de pasar por una funci\'on de activaci\'on, la cual es una multiplicaci\'on 
entre lo valores otorgados, al ser procesados por las capas que contenga la neurona, se obtendr\'a un valor distinto al de entrada.

Adem\'as son una distribuci\'on muy conocida de parte del Machine Learning, de otra manera, es el poder que tienen las computadoras para una buena estructura distribuida en paralelo y una buena habilidad de aprendizaje, 
este modelo computacional se define por medio de las neuronas biológicas las que son encargadas de que el ser humano pueda aprender o distinguir de distintos aspectos, este tipo de 
método es motivado para poder obtener la meta de un buen aprendizaje de m\'aquina \cite{liu2015}.

Un factor importante para esta rama es la perdida de memoria, este es un problema biol\'ogico, el cual tanto afecta a los humanos como a las maquinas, es por eso que se han elaborado distintos
experimentos para poder combatir esta problemática.
Uno de  estos es el caso de John Bullinaria, quien maneja la arquitectura de doble peso, ya que en su experimento da a notar que mejora la utilizaci\'on del aprendizaje incremental, esto se logr\'o 
con sistemas existentes como lo es Learn++.

Cabe mencionar que el experimento realizado fue un problema de generalizaci\'on m\'as comunes, pero se necesitan m\'as evidencias 
de que usando esta metodolog\'ia sirve para utilizarlo no solamente en tareas generalizadas, adem\'as se espera un mejor rendimiento \cite{Bullinaria2009}.  \\


