\section{Introduccion}

El aprendizaje incremental es un método el cual a sido implementado en el área de la inteligencia artificial, ya que al realizar tareas especificas de dicha rama nos 
ayuda a optimizarla para que el algoritmo sea más eficiente.
Cualquier tipo de aprendizaje puede ser considerado aprendizaje incremental si el problema a resolver tiene un entrenamiento simple, adem\'as este tipo de algoritmo es conocido como \textit{algoritmos lineales sin memoria},
en la mayor\'ia de los casos este tipo de aprendizaje es el preferido o favorito por los desarrolladores \cite{GiraudCarrier2000}.

Se requiere el uso de las redes neuronales para poder dar predicciones de sucesos los cuales pasan en nuestras vidas, al momento de decir predicción lo primero que se 
nos viene a la mente es la astrología, pero no todo es astrología, nosotros tenemos el poder de realizar predicciones con el uso de tecnología. Algunos de los grupos de estudios de inteligencia artificial de Microsoft
son conocidos como \textbf{AIWizzard}, ya que ellos son los encargados de fabricar magia.

Las redes neuronales son una distribuci\'on muy conocida de parte del Machine Learning, de otra manera es el poder que tienen las computadoras para una buena estructura distribuida en paralelo y una buena habilidad de aprendizaje, 
este modelo computacional se define por medio de las neuronas biologicas las que son encargadas de que el ser humano pueda aprender o distinguir de distintos aspectos, este tipo de 
metodo es motivado para poder obtener la meta de un buen aprendizaje de maquina \cite{liu2015}.

Otro factor importante que fluye en este proceso es la perdida de memoria a corto o largo plazo que pueden sufrir los algoritmos de machine learning, pero como se ha explicado las redes neuronales artificiales o por sus siglas en ingles 
\textit{ANN} se basan en las redes neuronales biologicas, esto quiere decir que la perdida de memoria a corto plazo de loa algoritmos se basan en el factor humano y esta problematica es una deficis muy com\'un, generalmente aparecen en pacientes
que tiene epilepsia, ya que la fisonomia de su cerebro esta involucrado en la incautacion \cite{TRAMONINEGRE2017}.

\section{Planeamiento}

Al momento de realizar proyectos donde se utilizan las redes neuronales para predicciones, necesitamos darle datos para que esta pueda ser entrenada y como tal nos de a nosotros un  valor lo mayor aproximado, pero no se puede quedar solo con unos datos de entrenamiento
ya que los datos van cambiando y si se siguen manejando con datos pasados, las predicciones que se deben realizar ser\'an erroneas, as\'i que si se le dan m\'as datos de entrenamiento a una red neural que ya esta programda de una manera, esta puede que colapse, pero si se
usa el aprendizaje incrementado este problema se puede diluir y hacer la que predicciones sean m\'as certeras sin tener alg\'un error duranrte los años.