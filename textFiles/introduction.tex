\section{Introduccion}

La inteligencia artificial (IA) es una área del conocimeinto que se enfoca en poder hacer máquinas que tengan un comportamiento y razonamiento humano, para que en un momento,  podamos interactuar con ellas sin darnos cuenta que estamo interactuando con una máquima. Así mismo,  también es posible pensar que mucho del desarrollo en el área de inteligenci artificial, es el poder tener mejores herramientas que nos ayuden a las actividades diarias. 

En este sentido, un área de la IA es la llamada Aprendizaje Máquina,  donde se estudian algoritmos que permitan aprender de forma automática una tarea. Así, una de las técnicas más conocidas en la actualidad, dentro del área de IA son las Redes Neuronales Artificiales (RNAs), siendo técnicas que realizan procesos matem\'aticos para poder aprendenderse tareas a resolver.  Algunas área en las que son útiles las RNAs son en el aprendizaje de tareas no lineales como predicicón, como la predicicón la capacidad de la red 5G, basada en el tráfico diario de este \cite{zhao2022} o clasificación e.g la clasificación de metales y rocas por medio de RNAs y lógica difusa \cite{salazar2013}. 

Las RNAs estan formadas por neuronas artificiales que simulan a las biologicas. Así los procesos quimicos que suceden en el cerebro, se simulan computacionalmente a través de señales que viajen a traves de las neuronals artificiales, de aquí en adelante simplemente se referira a ellas como ''neuronas``. Las neuronas en una RNA cuentan con una estructura distribuida en paralelo, presentando una buena habilidad de aprendizaje \cite{liu2015}.

Dentro del apredizaje máquina, cuando una técnica, e.g. RNA, se enfoca en aprender una tarea, se le conoce como \textit{algoritmo lineal sin memoria} siendo uno de los métodos más empleados desde el inicio de las RNAs \cite{GiraudCarrier2000}.  Sin emabrgo, si es necesario incorporar nueva información del problema,  es necesario volver a entrenar todo el modelo, considerando toda la información existente, i.e. la anterior y la nueva que acaba de llegar, es ahí donde nace el concepto de Aprendizaje Incremental, siendo un área enfocada en poder incorporar información unev adel problema en cuestión, sin tener que volver a re-entrenar todo el modelo.


Derivado del aprendizaje incremental se desprende el concepto de memoria dentro de la IA, analizando como una algoritmo de aprendizaje máquina puede olvidar la informción que se uso en un entrenamiento previo, al entrenar con informaicón más reciente. Si se hace la analogía con los humanos la memoría, es un factor importante para estudiar considernado la perdida de información aprendida, así este es problema biol\'ogico, el cual tanto afecta a los humanos como a las maquinas. Por ello, se han elaborado distintos experimentos para poder combatir esta problemática.  Uno de estos es el caso de \cite{Bullinaria2009},  donde se propone el manejo de RNAs con pesos dobles,  donde la primer capa de pesos esta enfocada a comportarse como memoria a carto plazo, y la segunda como memoria a largo plazo.  Los experimentos mostrados en \cite{Bullinaria2009} permiten notar un mejora en tareas de aprendizaje incremental, teniendo menos perdia de información en comparación de implementción anteriores como el algoritmo  Learn++ \cite{li2008,Elwell2011}.

Así, el presente trabajo de investigación esta enfocado en poder explorar nuevas configuraciones de pesos duplicados para poder extender el trabajo previamente presentado en \cite{Bullinaria2009}.







