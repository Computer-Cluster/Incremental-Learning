\section{Marco Teorico}

    \subsection{Revisión de la literatura}
        El humano tiene una forma de aprendizaje muy particular, la cual se basa del estudio, donde lee, escribe y practica acerca de
        su tema de interes, pero dicho aprendizaje se puede ir olvidando, esto es una acción muy común que a cualquier persona.
        Existen estudios donde se comenta que existen tres motivos del porque se olvidan las cosas, proviene parte de la regularización de las emociones,
        el como se adquirierón los conocimientos y porque el olvido es un proceso por el cual el ser humano transita a lo largo de su vida \cite{Nrby2015}. Pero cabe
        mencionar que esto no es lo único que causa la perdida de memoria, ya que existe la déficits de memoria. 

    \subsection{Aprendizaje Incremental}
        Con el pasar de los años la tecnología a evolucionado, eso quiere decir que el Aprendizaje Automático se ha actualizado, :( que la 
        cantidad de datos va aumentado con más frecuencia y a los datos no se les da tanta importacia, es en este momento donde se implementa este tipo de algoritmo,
        se puede utilizar el aprendizaje incremental en modelos donde no existen datos registrados, se usan algortimos de dicho indole  :( T-T
