\section{Marco Teorico}

    \subsection{Revisión de la literatura}
        El humano tiene una forma de aprendizaje muy particular, la cual se basa del estudio, donde lee, escribe y practica acerca de
        su tema de interes, pero dicho aprendizaje se puede ir olvidando, esto es una acción muy común que a cualquier persona.
        Existen estudios donde se comenta que existen tres motivos del porque se olvidan las cosas, proviene parte de la regularización de las emociones,
        el como se adquirierón los conocimientos y porque el olvido es un proceso por el cual el ser humano transita a lo largo de su vida \cite{Nrby2015}. Pero cabe
        mencionar que esto no es lo único que causa la perdida de memoria, ya que existe la déficits de memoria. 

    \subsection{Aprendizaje Humano}
        Al momento de hablar del aprendizaje humano, se debe de hablar de la ciencia cognitiva, que es quien se encarga de descubrir esta incognita,
        esta cienca lo estudia de un modo multidisiplinario, el cual abarca las \'areas de \cite{bransford2000}: 
        \begin{itemize}
            \item La antropología.
            \item La lingüística.
            \item La filosofía.
            \item La sicología del desarrollo.
            \item La ciencia de la computación. 
            \item La neurociencia.
        \end{itemize}
        Con el metodo de esta cienca podemos descubrir dos tipos de aprendizaje que son:
        \begin{enumerate}
            \item Aprendizaje con Compresi\'on.
            \item Aprendizaje Activo.
        \end{enumerate}
        \subsubsection{Aprendizaje con Compresi\'on}
            La comprensi\'on es una activiadad la cual se ha generado al momento de realizar cualquier tipo de lectura.\\
            Al hablar de este tema nos enfocamos en el \'ambito estudiantil que es donde m\'as se maneja esta t\'actica, esto es una
            practica algo compleja, sist\'ematica y organizada, ya que nos da el significado de la literatura, gracias a esto se puede
            obtener el contexto de la literatura.

            Al conocer esto podemos decir con seguridad que para cualquier tipo de aprendizaje la comprensi\'on es 
            una parte primordial \cite{perez2014}.

        \subsubsection{Aprendizaje Activo}
            El aprendizaje de la forma en la que la conocemos no es del todo efectiva, se dice esto porque parte del sistema de educaci\'on porque la manera 
            correcta es con el principio de \textit{belongingness} el cual esta asociado al estimulo con su respuesta
            y esto es lo m\'as importante para que el ser humano pueda aprender cualquier cosa.\\
            Este tipo de aprendizaje se basa en la recepci\'on de conocimientos y la pr\'actica donde se ponen en marcha los conocimientos adquiridos.\\
            Otro concepto importante aqu\'i es la tautolog\'ia doble (\textit{selbstt\"atiges Lernen}) que en palabras informales es convertirse en autodidacta, 
            podemos observar que esto pertener a dicho de aprendizaje porque usa el principio mencionado anteriormente \cite{Huber2008}.



    \subsection{Aprendizaje Incremental}
        Con el pasar de los años la tecnología a evolucionado, eso quiere decir que el Aprendizaje Automático se ha actualizado, que la 
        cantidad de datos va aumentado con más frecuencia y a los datos no se les da tanta importacia, es en este momento donde se implementa 
        este tipo de algoritmo,  
